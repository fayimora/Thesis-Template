%
% A header that lets you compile a chapter by itself, or inside a larger document.
% Adapted from http://stackoverflow.com/a/3655553/751077
% Use \inbpdocument and \outbpdocument in your individual files, in place of \begin{document} and \end{document}.
% In your main file, put in a \def \ismaindoc {} before including or importing anything.


\ifx\ismaindoc\undefined
    \newcommand{\inbpdocument}{
        \def \ismaindoc {}
        % Use this header if we are compiling by ourselves.
        \documentclass[a4paper,12pt,twoside]{report}
        
\usepackage{graphicx}
\usepackage{verbatim}
\usepackage{latexsym}
\usepackage{mathchars}
\usepackage{setspace}
\usepackage[left=2cm,right=2cm,top=2cm,bottom=3cm]{geometry}
\usepackage[inline]{enumitem}
\usepackage[utf8]{inputenc}
\usepackage{hyperref}
\usepackage{natbib}
\usepackage{authordate1-4}
\usepackage{appendix}
\usepackage[nottoc]{tocbibind} % add Bibliography to toc
\usepackage{caption}
\usepackage{subcaption}
\usepackage{booktabs}
\usepackage{longtable}
\usepackage{algorithm2e}
\usepackage{float}
\usepackage{amsmath}
% \restylefloat{table}

\hypersetup{linktoc=all}
% \usepackage{url}

\setcounter{secnumdepth}{4}
\setcounter{tocdepth}{4}


\setlength{\parskip}{\medskipamount}  % a little space before a \par
\setlength{\parindent}{0pt}	      % don't indent first lines of paragraphs

%UHEAD.STY  If this is included after \documentstyle{report}, it adds
% an underlined heading style to the LaTeX report style.
% \pagestyle{uheadings} will put underlined headings at the top
% of each page. The right page headings are the Chapter titles and
% the left page titles are supplied by \def\lefthead{text}.

% Ted Shapin, Dec. 17, 1986

\makeatletter
\def\chapapp2{Chapter}

% \def\appendix{\par
%  \setcounter{chapter}{0}
%  \setcounter{section}{0}
%  \def\chapapp2{Appendix}
%  \def\@chapapp{Appendix}
%  \def\thechapter{\Alph{chapter}}}

\def\ps@uheadings{\let\@mkboth\markboth
% modifications
\def\@oddhead{\protect\underline{\protect\makebox[\textwidth][l]
		{\sl\rightmark\hfill\rm\thepage}}}
\def\@oddfoot{}
\def\@evenfoot{}
\def\@evenhead{\protect\underline{\protect\makebox[\textwidth][l]
		{\rm\thepage\hfill\sl\leftmark}}}
% end of modifications
\def\chaptermark##1{\markboth {\ifnum \c@secnumdepth >\m@ne
 \chapapp2\ \thechapter. \ \fi ##1}{}}%
\def\sectionmark##1{\markright {\ifnum \c@secnumdepth >\z@
   \thesection. \ \fi ##1}}}
\makeatother



%%From: marcel@cs.caltech.edu (Marcel van der Goot)
%%Newsgroups: comp.text.tex
%%Subject: illegal modification of boxit.sty
%%Date: 28 Feb 92 01:10:02 GMT
%%Organization: California Institute of Technology (CS dept)
%%Nntp-Posting-Host: andromeda.cs.caltech.edu
%%
%%
%%Quite some time ago I posted a file boxit.sty; maybe it made it
%%to some archives, although I don't recall submitting it. It defines
%%	\begin{boxit}
%%	...
%%	\end{boxit}
%%to draw a box around `...', where the `...' can contain other
%%environments (e.g., a verbatim environment). Unfortunately, it had
%%a problem: it did not work if you used it in paragraph mode, i.e., it
%%only worked if there was an empty line in front of \begin{boxit}.
%%Luckily, that is easily corrected.
%%
%%HOWEVER, apparently someone noticed the problem, tried to correct it,
%%and then distributed this modified version. That would be fine with me,
%%except that:
%%1. There was no note in the file about this modification, it only has my
%%   name in it.
%%2. The modification is wrong: now it only works if there is *no* empty
%%   line in front of \begin{boxit}. In my opinion this bug is worse than
%%   the original one.
%%
%%In particular, the author of this modification tried to force an empty
%%line by inserting a `\\' in the definition of \Beginboxit. If you have
%%a version of boxit.sty with a `\\', please delete it. If you have my
%%old version of boxit.sty, please also delete it. Below is an improved
%%version.
%%
%%Thanks to Joe Armstrong for drawing my attention to the bug and to the
%%illegal version.
%%
%%                                          Marcel van der Goot
%% .---------------------------------------------------------------
%% | Blauw de viooltjes,                    marcel@cs.caltech.edu
%% |    Rood zijn de rozen;
%% | Een rijm kan gezet
%% |    Met plaksel en dozen.
%% |


% boxit.sty
% version: 27 Feb 1992
%
% Defines a boxit environment, which draws lines around its contents.
% Usage:
%   \begin{boxit}
%	... (text you want to be boxed, can contain other environments)
%   \end{boxit}
%
% The width of the box is the width of the contents.
% The boxit* environment behaves the same, except that the box will be
% at least as wide as a normal paragraph.
%
% The reason for writing it this way (rather than with the \boxit#1 macro
% from the TeXbook), is that now you can box verbatim text, as in
%   \begin{boxit}
%   \begin{verbatim}
%   this better come out in boxed verbatim mode ...
%   \end{verbatim}
%   \end{boxit}
%
%						Marcel van der Goot
%						marcel@cs.caltech.edu
%

\def\Beginboxit
   {\par
    \vbox\bgroup
	   \hrule
	   \hbox\bgroup
		  \vrule \kern1.2pt %
		  \vbox\bgroup\kern1.2pt
   }

\def\Endboxit{%
			      \kern1.2pt
		       \egroup
		  \kern1.2pt\vrule
		\egroup
	   \hrule
	 \egroup
   }	

\newenvironment{boxit}{\Beginboxit}{\Endboxit}
\newenvironment{boxit*}{\Beginboxit\hbox to\hsize{}}{\Endboxit}

\pagestyle{empty}

\setlength{\parskip}{2ex plus 0.5ex minus 0.2ex}
\setlength{\parindent}{0pt}

\makeatletter  %to avoid error messages generated by "\@". Makes Latex treat "@" like a letter

\linespread{0.66}
\def\submitdate#1{\gdef\@submitdate{#1}}
\def\supervisor#1{\gdef\@supervisor{Supervisor: #1}}

\def\maketitle{
  \begin{titlepage}{
    %\linespread{1.5}
    \Large Queen Mary, University of London \\
    %\linebreak
    Department of Electronic Engineering and Computer Science
    \rm
    \vskip 3in
    \Large \bf \@title \par
  }
  \vskip 0.3in
  \par
  {\Large \@author}
  \vskip 0.1in
  {\Large \@supervisor}

  \vskip 4in
  \par
  Submitted in part fulfilment of the requirements for the degree of
  \linebreak
  BSc Computer Science with Industrial Experience, \@submitdate
  \vfil
  \end{titlepage}
}

\def\titlepage{
  \newpage
  \centering
  \linespread{1}
  \normalsize
  \vbox to \vsize\bgroup\vbox to 9in\bgroup
}
\def\endtitlepage{
  \par
  \kern 0pt
  \egroup
  \vss
  \egroup
  \cleardoublepage
}

\def\abstract{
  \begin{center}{
    \large\bf Abstract}
  \end{center}
  \small
  %\def\baselinestretch{1.5}
  \linespread{1.5}
  \normalsize
}
\def\endabstract{
  \par
}

\newenvironment{acknowledgements}{
  \cleardoublepage
  \begin{center}{
    \large \bf Acknowledgements}
  \end{center}
  \small
  \linespread{1.5}
  \normalsize
}{\cleardoublepage}
\def\endacknowledgements{
  \par
}

\newenvironment{dedication}{
  \cleardoublepage
  \begin{center}{
    \large \bf Dedication}
  \end{center}
  \small
  \linespread{1.5}
  \normalsize
}{\cleardoublepage}
\def\enddedication{
  \par
}

\def\preface{
    \pagenumbering{roman}
    \pagestyle{plain}
    \doublespacing
}

\def\body{
    \cleardoublepage
    \pagestyle{uheadings}
    \singlespacing
    \tableofcontents
    \doublespacing
    \normalsize
    \pagestyle{plain}

    \cleardoublepage
    \pagestyle{uheadings}
    \listoftables
    \pagestyle{plain}

    \cleardoublepage
    \pagestyle{uheadings}
    \listoffigures
    \pagestyle{plain}

    \cleardoublepage
    \pagestyle{uheadings}
    \pagenumbering{arabic}
    % \doublespacing
}

\makeatother  %to avoid error messages generated by "\@". Makes Latex treat "@" like a letter


\newcommand{\ipc}{{\sf ipc}}

\newcommand{\Prob}{\bbbp}
\newcommand{\Real}{\bbbr}
\newcommand{\real}{\Real}
\newcommand{\Int}{\bbbz}
\newcommand{\Nat}{\bbbn}

\newcommand{\NN}{{\sf I\kern-0.14emN}}   % Natural numbers
\newcommand{\ZZ}{{\sf Z\kern-0.45emZ}}   % Integers
\newcommand{\QQQ}{{\sf C\kern-0.48emQ}}   % Rational numbers
\newcommand{\RR}{{\sf I\kern-0.14emR}}   % Real numbers
\newcommand{\KK}{{\cal K}}
\newcommand{\OO}{{\cal O}}
\newcommand{\AAA}{{\bf A}}
\newcommand{\HH}{{\bf H}}
\newcommand{\II}{{\bf I}}
\newcommand{\LL}{{\bf L}}
\newcommand{\PP}{{\bf P}}
\newcommand{\PPprime}{{\bf P'}}
\newcommand{\QQ}{{\bf Q}}
\newcommand{\UU}{{\bf U}}
\newcommand{\UUprime}{{\bf U'}}
\newcommand{\zzero}{{\bf 0}}
\newcommand{\ppi}{\mbox{\boldmath $\pi$}}
\newcommand{\aalph}{\mbox{\boldmath $\alpha$}}
\newcommand{\bb}{{\bf b}}
\newcommand{\ee}{{\bf e}}
\newcommand{\mmu}{\mbox{\boldmath $\mu$}}
\newcommand{\vv}{{\bf v}}
\newcommand{\xx}{{\bf x}}
\newcommand{\yy}{{\bf y}}
\newcommand{\zz}{{\bf z}}
\newcommand{\oomeg}{\mbox{\boldmath $\omega$}}
\newcommand{\res}{{\bf res}}
\newcommand{\cchi}{{\mbox{\raisebox{.4ex}{$\chi$}}}}
%\newcommand{\cchi}{{\cal X}}
%\newcommand{\cchi}{\mbox{\Large $\chi$}}

% Logical operators and symbols
\newcommand{\imply}{\Rightarrow}
\newcommand{\bimply}{\Leftrightarrow}
\newcommand{\union}{\cup}
\newcommand{\intersect}{\cap}
\newcommand{\boolor}{\vee}
\newcommand{\booland}{\wedge}
\newcommand{\boolimply}{\imply}
\newcommand{\boolbimply}{\bimply}
\newcommand{\boolnot}{\neg}
\newcommand{\boolsat}{\!\models}
\newcommand{\boolnsat}{\!\not\models}


\newcommand{\op}[1]{\mathrm{#1}}
\newcommand{\s}[1]{\ensuremath{\mathcal #1}}

% Properly styled differentiation and integration operators
\newcommand{\diff}[1]{\mathrm{\frac{d}{d\mathit{#1}}}}
\newcommand{\diffII}[1]{\mathrm{\frac{d^2}{d\mathit{#1}^2}}}
\newcommand{\intg}[4]{\int_{#3}^{#4} #1 \, \mathrm{d}#2}
\newcommand{\intgd}[4]{\int\!\!\!\!\int_{#4} #1 \, \mathrm{d}#2 \, \mathrm{d}#3}

% Large () brackets on different lines of an eqnarray environment
\newcommand{\Leftbrace}[1]{\left(\raisebox{0mm}[#1][#1]{}\right.}
\newcommand{\Rightbrace}[1]{\left.\raisebox{0mm}[#1][#1]{}\right)}

% Funky symobols for footnotes
\newcommand{\symbolfootnote}{\renewcommand{\thefootnote}{\fnsymbol{footnote}}}
% now add \symbolfootnote to the beginning of the document...

\newcommand{\normallinespacing}{\renewcommand{\baselinestretch}{1.5} \normalsize}
\newcommand{\mediumlinespacing}{\renewcommand{\baselinestretch}{1.2} \normalsize}
\newcommand{\narrowlinespacing}{\renewcommand{\baselinestretch}{1.0} \normalsize}
\newcommand{\bump}{\noalign{\vspace*{\doublerulesep}}}
\newcommand{\cell}{\multicolumn{1}{}{}}
\newcommand{\spann}{\mbox{span}}
\newcommand{\diagg}{\mbox{diag}}
\newcommand{\modd}{\mbox{mod}}
\newcommand{\minn}{\mbox{min}}
\newcommand{\andd}{\mbox{and}}
\newcommand{\forr}{\mbox{for}}
\newcommand{\EE}{\mbox{E}}

\newcommand{\deff}{\stackrel{\mathrm{def}}{=}}
\newcommand{\syncc}{~\stackrel{\textstyle \rhd\kern-0.57em\lhd}{\scriptstyle L}~}

\def\coop{\mbox{\large $\rhd\!\!\!\lhd$}}
\newcommand{\sync}[1]{\raisebox{-1.0ex}{$\;\stackrel{\coop}{\scriptscriptstyle
#1}\,$}}

\newtheorem{definition}{Definition}[chapter]
\newtheorem{theorem}{Theorem}[chapter]

\newcommand{\Sectionref}[1]{Section~\ref{#1}}
\newcommand{\Tableref}[1]{Table~\ref{#1}}
\newcommand{\Pageref}[1]{page~\pageref{#1}}

\newcommand{\Figref}[1]{Figure~\ref{#1}}
\newcommand{\fig}[3]{
 \begin{figure}[!ht]
 \begin{center}
 \scalebox{#3}{\includegraphics{figs/#1.ps}}
 \vspace{-0.1in}
 \caption[ ]{\label{#1} #2}
 \end{center}
 \end{figure}
}

\newcommand{\figtwo}[8]{
 \begin{figure}
 \parbox[b]{#4 \textwidth}{
 \begin{center}
 \scalebox{#3}{\includegraphics{figs/#1.ps}}
 \vspace{-0.1in}
 \caption{\label{#1}#2}
 \end{center}
 }
 \hfill
 \parbox[b]{#8 \textwidth}{
 \begin{center}
 \scalebox{#7}{\includegraphics{figs/#5.ps}}
 \vspace{-0.1in}
 \caption{\label{#5}#6}
 \end{center}
 }
 \end{figure}
}

        \normallinespacing
        \begin{document}
    }
    \newcommand{\outbpdocument}[1]{
      % Fake chapters so references aren't broken
        \label{cha:introduction}
        \label{cha:background}
        \label{cha:data-classification}
        \label{cha:topic-modelling}

        \bibliographystyle{authordate1}
        \bibliography{bibliography}
        \end{document}
    }
\else
    %If we're inside another document, no need to re-start the document.
    \ifx\inbpdocument\undefined
        \newcommand{\inbpdocument}{}
        \newcommand{\outbpdocument}[1]{}
    \fi
\fi

\inbpdocument

\chapter{Introduction}
\label{cha:introduction}

\section{Motivation}
\label{sec:motivation}
Organisations today continuously search for new ways to get feedback from their clients in a bid to
improve customer satisfaction. Technology firms like Apple, Samsung and Google want to know if their
software/hardware products meet their consumers' needs. Merchandise retailers like Walmart and Tesco
are constantly trying to make sure they are serving the right products in the right quantity and at
for right price. Startups continuously evaluate their products to measure the probability of the
company being successful sometime in the future. Postal services like Royal Mail are very interested
in how their services are doing and what their customers despise most so they can improve.

Current ways of achieving this include \textbf{Surveys} (questionnaires or interviews) and
\textbf{Focus Groups}.  Surveys are very easy to create and distribute. There are also a variety of
tools to help with this such as SurveyMonkey\footnote{https://www.surveymonkey.com/} and Google
Docs\footnote{https://drive.google.com}. Unfortunately, Surveys also have a few unpleasant drawbacks
like time consumption and labour intensity. It can also be difficult to encourage participants to
respond. Nevertheless, the main drawback to using Surveys is that some questions are left unanswered
while the answers given in answered questions may not reflect the truthful sentiments of the
participant.~\citet{rubin1987} concurs with this and he goes on to discuss how this problem can be
solved (to a certain extent) with imputation\footnote{Imputation is the process of inferring
plausible values for missing entries}.~\citet{hayes2008} also agrees with this point of view and
suggests the use of well designed leading questions to put the participant in the right frame of
mind. For instance, a leading question like ``\textit{How likely will you recommend our service to
friends?}'' gets the participant thinking about recommendations. While the above solutions might
work, they have the same drawbacks as the original problem. Imputation can be very time consuming,
labour intensive and error prone while the use of leading questions fails to solve the problem of
unanswered questions.

Unfortunately, interviews and focus groups also suffer from false answers due to the fact that they
are not anonymous. This means that the participants, in the face of an interviewer, try to be
lenient in other not to sound too negative. This could sometimes be due to the fact that
participation in the interview/focus group has been incentivised with money or desirable items.

Ideally, the next question we should be asking is ``\textit{How can we get the truthful views of our
clients about our products and services}''? We need to find a way to get this information without
putting any pressure on our clients.

\section{Aims and Objectives}
\label{sec:objectives}
The aim of this project is to investigate other means of getting our customer views and also,
how we can make use of Machine Learning and Natural Language Processing techniques to make sense of
the data.

Fortunately, the recent surge in the use of social media makes the former relatively easy. People,
more often than not, tend to post their truthful feelings about services they use on social media.
For instance, Person A buys an iPhone today and realises that the Wi-Fi connectivity is faulty.
He/She will most likely post something like ``\textit{New iPhone wifi not working \#NotCool}'' on
one or more of the available social networking platforms. From this statement, we can infer that
Person A is talking about \textit{the iPhone}, \textit{Wi-Fi} and \textit{Connectivity}. The process
of discovering abstract topics in text is called \textbf{Topic Modelling} and
Chapter~\ref{cha:topic-modelling} discusses how we can automate this process.

We try to answer two main research questions. They include:
\begin{itemize}
  \item Can we use supervised techniques to accurately classify tweets into what is relevant and
    what is not?
  \item Can we detect themes/topics in our dataset? If yes, are these topics related to Apple Inc in
    any way?
\end{itemize}


\section{Why Twitter?}
\label{sec:why-twitter}
Twitter is a social micro-blogging platform where users can share messages in 140 characters. It
also allows its users to follow each other. This means, if person A follows person B, A will see
public messages from B. These messages are usually referred to as tweets.

Tweets are capped to 140 characters and can contain text, links or a combination of both. They are
usually related to either an event, interests or just personal opinion. Facebook posts are mostly
always well thought out and each post might include multiple topics. Tweets on the other hand are
usually written at the speed of thought.

According to Mashable, DOMO, a Business Intelligence company paired up with Column Five Media to
create an infographic\footnote{See \url{http://mashable.com/2012/06/22/data-created-every-minute/}}
about the web back in 2012. It showed that Twitter at the time received around 100,000 tweets per
minute. As at 1st February 2014 Twitter claims to receive 500 million tweets a day\footnote{See
https://about.twitter.com/company}. That is roughly 350,000 tweets per minute which is over 3 times
the amount 2 years before. Twitter also claims to have 241 million monthly users.

Finally, Twitter's data is open compared to other social platforms like Facebook. This means
developers are free to tap into this wealth of data in almost real time and free of charge. This
makes Twitter a very good source for our data.


\section{Methodology}
\label{sec:methodology}
This study requires social data and the dataset used was gathered from Twitter between October and
November 2013. Each tweet in the dataset is in someway related to Apple Inc and/or their products.

We then train a classifier to help filter out as many irrelevant tweets as possible. We briefly
analyse different ways to filter the dataset but eventually settle with using Na\"{i}ve Bayes
Classifier. We also look into different ways of analysing the classifier's performance and ways it
can be improved.

Finally, we attempt to identify topics/themes in the dataset. We briefly look at Latent Semantic
Indexing and why it might not be suitable for our needs. We then look into Latent Dirichlet
Allocation, a common approach to topic modelling and use it to detect topics in our dataset. The
evaluation of topics generated will be analysed empirically and qualitatively. This means we take a
topic and make some assumptions about the semantics of the tweets belonging to that topic. We then
analyse the tweets to confirm the validity of our assumption.

\section{Statement of Originality}
This report with any accompanying implementation, is submitted as part requirement for the degree of
Computer Science with Industrial Experience at Queen Mary, University of London. I certify that it
has not been submitted for any degree or other purposes.

I certify that the intellectual content of this report, to the best of my knowledge, is the product
of my own labour except where indicated in the text.

% To the best of my knowledge, the content of this report and any accompanying implementation is the
% product of my own labour except where indicated in the text.

\outbpdocument{
    \bibliographystyle{authordate1}
    \bibliography{bibliography}
}
